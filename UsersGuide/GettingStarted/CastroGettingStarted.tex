
\section{Downloading the Code}

CASTRO is built on top of the BoxLib framework.  In order to run CASTRO, you must download
two separate git modules.

\vspace{.1in}

\noindent First, make sure that {\tt git} is installed on your machine---we recommend version 1.7.x or higher.

\vspace{.1in}

\begin{enumerate}

\item Download the BoxLib repository by typing: 

\noindent {\tt git clone https://ccse.lbl.gov/pub/Downloads/BoxLib.git}

\noindent This will create a directory called {\tt BoxLib/} on your
machine.  Put this somewhere out of the way and set the environment
variable, {\tt BOXLIB\_HOME}, on your machine to the path name where
you have put BoxLib.  You will want to periodically update BoxLib by
typing

\noindent {\tt git pull}

in the BoxLib directory.  

\item Now download the CASTRO repository by typing 

\noindent {\tt git clone https://ccse.lbl.gov/pub/Downloads/Castro.git }

\end{enumerate}

%\clearpage

\section{Building the Code}

\begin{enumerate}

\item From the directory in which you checked out the Castro git repo, type

{\tt cd Castro/Exec/Sedov}

This will put you into a directory in which you can run the Sedov problem in 1-d, 2-d or 3-d.
\item In {\tt Sedov/}, edit the {\tt GNUmakefile}, and set

{\tt DIM = 2} (for example)

{\tt COMP = gcc} (or your favorite C++ compiler)

{\tt FCOMP = gfortran} (your favorite Fortran 90 compiler)

{\tt DEBUG = FALSE}

If you want to try other compilers than the GNU suite and they don't work, 
please let us know.  

To build a serial (single-processor) code, set {\tt USE\_MPI = FALSE}.
This will compile the code without the MPI library.  If you want to do
a parallel run, then you would set {\tt USE\_MPI = TRUE}.  In this
case, the build system will need to know about your MPI installation.
This can be done by editing the makefiles in the BoxLib tree, but a
simple method is to set the shell environment variable {\tt
  BOXLIB\_USE\_MPI\_WRAPPERS=1}.  If this is set, then the build
system will fall back to using the local MPI compiler wrappers
(e.g.\ {\tt mpic++} and {\tt mpif90}) to do the build.

\item Now type {\tt make}. The resulting executable will look something like 
{\tt Castro2d.Linux.gcc.gfortran.ex}, which means this is a 2-d version of the code, 
made on a Linux machine, with {\tt COMP = gcc} and {\tt FCOMP = gfortran}.

\end{enumerate}

\section{Running the Code}

\begin{enumerate}

\item Type:

\noindent {\tt Castro2d.Linux.gcc.gfortran.ex inputs.2d.cyl\_in\_cartcoords} 

\noindent This will run the 2-d cylindrical Sedov problem in Cartesian (x-y coordinates). 
You can see other possible options, which should be clear by the names of the inputs files.

\item You will notice that running the code generates directories that look like 
{\tt plt00000}, {\tt plt00020}, etc, and {\tt chk00000}, {\tt chk00020}, etc. These are "plotfiles" and 
"checkpoint" files. The plotfiles are used for visualization, the checkpoint files are 
used for restarting the code.

\end{enumerate}

\section{Visualization of the Results}

\begin{enumerate}

\item To visualize the plotfiles, you can use a freely available visualization package
like VisIt, or you can try ``Amrvis.'' To get Amrvis, type

\noindent {\tt git clone https://ccse.lbl.gov/pub/Downloads/Amrvis.git}

\noindent Then cd into {\tt Amrvis/}, edit the {\tt GNUmakefile} there
to set {\tt DIM = 2}, and again set {\tt COMP} and {\tt FCOMP} to compilers that
you have. Leave {\tt DEBUG = FALSE}. Then type {\tt make}.  This will make an
executable that looks like {\tt amrvis2d...ex}.

If you want to build amrvis with {\tt DIM = 3}, you must first download and build {\tt volpack}.  Type

\noindent {\tt git clone https://ccse.lbl.gov/pub/Downloads/volpack.git}

\noindent Then cd into {\tt volpack/} and type {\tt make}.

\noindent Note: This requires the OSF/Motif libraries and headers. If you don't have these 
you will need to install the development version of motif through your package manager. 
{\tt lesstif} gives some functionality and will allow you to build the amrvis executable, 
but amrvis will not run properly.

\noindent Note: On most Linux distributions, motif library is provided
by the openmotif package, and its header files (like {\tt Xm.h}) are
provided by openmotif-devel. If those packages are not installed, then
use the package management tool to install them, which varies from
distribution to distribution, but is straightforward. I can provide
detailed instructions if anyone needs them.

You may then want to create an alias to amrvis2d, for example

\noindent {\tt alias amrvis2d /tmp/Amrvis/amrvis2d...ex}

\item Return to the {\tt Castro/Exec/Sedov} directory.  Type {\tt
  amrvis2d plt00152} to see a single plotfile, or {\tt amrvis2d -a
  plt*}, which will animate the sequence of plotfiles. Try playing
  around with this -- note you can change which variable you are
  looking at, you can select a region and click "Dataset" (under View)
  in order to look at the actual numbers, etc. You can also export the
  pictures in several different formats -- under "File", see "Export".

Please know that we do have a number of conversion routines to other
formats (such as matlab), but it is hard to describe them all. If you
would like to display the data in another format, please let us know
(again, asalmgren@lbl.gov) and we will point you to whatever we have
that can help.

\end{enumerate}

You have now completed a brief introduction to CASTRO. 
